% Universidade Aberta
% Template TeX para relatório de trabalhos
% 2024
%
%
% Dados para a capa
\newcommand{\Titulo}{Título}
\newcommand{\Ano}{2024}
\newcommand{\Autor}{Autor}
%
%
\documentclass[12pt,a4paper,final]{article}
\usepackage{csquotes}
\usepackage[portuguese]{babel}
\usepackage{polyglossia}
\setdefaultlanguage{portuguese}
\usepackage{graphicx}
\graphicspath{ {./images/} }
\usepackage[a4paper,top=3cm,bottom=3cm,left=3.5cm,right=2cm]{geometry}
\usepackage{booktabs}
\usepackage{fontspec,xltxtra,xunicode}
\defaultfontfeatures{Mapping=tex-text}
\setmainfont[Ligatures=TeX]{Times New Roman}
\setromanfont[Mapping=tex-text]{Times New Roman}
\usepackage[pdfauthor=\Autor,
            pdftitle=\Titulo, 
		    colorlinks=true,
            linkcolor=black,
            citecolor=black,
            bookmarksopen=true]{hyperref}
\hypersetup{colorlinks, citecolor=black, urlcolor=black}           
\usepackage{bookmark}
\usepackage[style=apa, backend=biber, sortcites, url=true]{biblatex}
\addbibresource{ref.bib}
\renewcommand{\baselinestretch}{1.5}
\begin{document}
\title{\Titulo}
\author{\Autor}
\date{\Ano}
\pagenumbering{gobble}
\begin{titlepage}
    \begin{center}
        \vspace*{4cm}
        
        \textbf{\large UNIVERSIDADE ABERTA}
        
        \textbf{\large UNIVERSIDADE DE TRÁS-OS-MONTES E ALTO DOURO}
        
        \vspace{1cm}
        
		\begin{minipage}{0.4\textwidth}
            \centering
            \includegraphics[width=0.8\textwidth]{uab}
        \end{minipage}
        \begin{minipage}{0.4\textwidth}
            \centering
            \includegraphics[width=0.8\textwidth]{utad}
        \end{minipage}
        
        \vspace{1.5cm}
        
        \textbf{\large \Titulo}
        
        \vspace{1.5cm}
        
        \textbf{\large \Autor}
        
        \vspace{2cm}
        
        \textbf{\large Mestrado em Engenharia Informática e Tecnologia Web}
        \vfill
        \textbf{\Ano}
    \end{center}
\end{titlepage}
\renewcommand{\contentsname}{Índice}
\cleardoublepage
\pagenumbering{roman}
\tableofcontents
\newpage
\listoffigures
\newpage
\listoftables
\newpage
\cleardoublepage
\pagenumbering{arabic}
\section{Introdução}
Este é o texto da introdução. 

\begin{figure}[h!]
    \centering
    \includegraphics[width=0.9\textwidth]{img_meitw}
    \caption{Exemplo de uma figura}
    \label{fig:exemplo}
\end{figure}

\section{Metodologia}
Este é o texto da metodologia.

\begin{enumerate}
    \item Item 1
    \item Item 2
    \item Item 3
\end{enumerate}

\section{Descrição}
Detalhes da descrição da metodologia.

\begin{itemize}
    \item Item 1
    \item Item 2
    \item Item 3
\end{itemize}

\section{Estado da Arte}
\subsection{Pesquisa}
Detalhes da pesquisa sobre o estado da arte.

\subsection{Critérios}
Critérios da pesquisa sobre o estado da arte.

\section{Resultados}
Este é o texto dos resultados.

\begin{table}[]
\centering
\resizebox{0.2\columnwidth}{!}{%
\begin{tabular}{@{}|c|cccc|@{}}
\toprule
  & A & B & C & D \\ \midrule
X & 4 & 3 & 2 & 1 \\ \midrule
Y & 2 & 3 & 4 & 5 \\ \midrule
Z & 4 & 3 & 2 & 1 \\ \bottomrule
\end{tabular}%
}
\caption{Tabela Exemplo}
\label{tab:my-table}
\end{table}

\section{Conclusões}
Este é o texto da conclusão com uma citação \cite{su15010857}.

\newpage
\printbibliography
\end{document}
